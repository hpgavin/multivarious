
\documentclass[11pt]{article}
\usepackage[utf8]{inputenc}
\usepackage[T1]{fontenc}
\usepackage{lmodern}
\usepackage{microtype}
\usepackage{amsmath,amssymb,amsfonts}
\usepackage{mathtools}
\usepackage{geometry}
\usepackage{hyperref}
\usepackage{enumitem}
\usepackage{fancyhdr}
\usepackage{setspace}
\usepackage{caption}
\usepackage{graphicx}

\geometry{letterpaper, margin=1in}
\hypersetup{
  colorlinks=true,
  linkcolor=blue,
  citecolor=blue,
  urlcolor=blue,
  pdftitle={Confidence Intervals for the Empirical CDF (ECDF)},
  pdfauthor={ChatGPT}
}

\pagestyle{fancy}
\fancyhf{}
\lhead{ECDF Confidence Intervals}
\rhead{\thepage}
\setlength{\headheight}{14pt}
\onehalfspacing

\begin{document}

\begin{center}
  {\LARGE \textbf{Confidence Intervals for the Empirical CDF (ECDF)}}\\[6pt]
  {\large Conversation: user's question and detailed explanation}\\[8pt]
  {\normalsize Generated for local compilation}
\end{center}

\vspace{12pt}

\section*{User's Original Question}

You wrote:

\begin{quote}
I am interested in gaining a better understanding of the statistical confidence intervals around an empirical cumulative distribution (ECDF). The ECDF of a sorted data sample
\[
x_{(1)} \le x_{(2)} \le x_{(3)} \le \cdots \le x_{(i-1)} \le x_{(i)} \le x_{(i+1)} \le \cdots \le x_{(N)}
\]
is \(F(x_{(i)}) = i/(N+1)\) according to the 1958 text by Emil Gumbel called \emph{Extreme Value Statistics}. The variance of \(F(x_{(i)})\) is
\[
V(F) = \frac{F(1 - F)}{N + 2}.
\]

So the variance tapers to zero at \(F=1\) and \(F=0\). The confidence interval would be
\[
F(x) \pm z_{\alpha/2}\sqrt{V(x)}.
\]


Three conundrums about this ... 
\begin{enumerate}
\item The expression  F(x) +/- z(alpha) * sqrt(V(x))  presumes that F(x) is normally distributed for any value of x, but F(x) must lie between 0 and 1, and so the confidence interval should be symmetric about F(x) only at $F = 1/2$.     So the normal distribution does not seem appropriate for this case.   
  
\item related to conundrum (1).    The confidence   $F(x) \pm z(\alpha)  \sqrt{V}$ must be bounded within $[0 , 1 ]$.   But for large values of $z$, it can exceed this range.      Note here thate the Dvoretzky-Kiefer-Wolfowitz (DKW) inequality bounds can also exceed the $[0,1]$ interval.     This issue ties to the variance of the ECDF V(F)  which certainly goes to zero as $F$ goes to zero or one.   

\item Why would the confidence interval on the ECDF not be larger at extreme values, where data is more sparse?    At extreme values the Gumble bound is tighter and the DKW bound is uniform.  

It seems that the standard normal variate is not approriate.  It seems that an appropriate distribution would be bounded within the domain $[0 < F < 1]$ and is symmetric at ($F = 1/2$) and has a two variables ... a mean (or median) $F$ and a variance $V$.    
\end{enumerate}

Question:   What is the appropriate distribution function for confidence intervals on the ECDF?  


\end{quote}

\section*{Detailed Answer}

\subsection*{Short Answer}

For finite-sample ECDF confidence intervals, the correct distributions are:

\begin{itemize}
  \item \textbf{Binomial} for fixed \(x\)
(you observe $k$ of $N$ samples $\leq x$): the ECDF value
$F^N(x)=k/N$ has an exact Binomial distribution: $k \sim Binomial(N,F(x))$.
Confidence intervals should be formed from the binomial distribution
(exact Clopper-Pearson) or good approximations (Wilson, Jeffreys/Beta)
These are automatically bounded in [0,1] and generally asymmetric near the edges.

  \item \textbf{Beta} for order statistics: \(U_i \sim \mathrm{Beta}(i, N+1-i)\)

The random variable $U_i=F(X(i))$ (the population CDF evaluated at the i-th order statistic) has the Beta distribution
\[  U_i \sim \mbox{Beta}(i,N+1-i), \]

with mean $i/(N+1)$ and variance $i(N+1-i) / ((N+1)^2(N+2))$. Use Beta quantiles for exact, bounded, asymmetric intervals around the plotting positions. 

These produce bounded, asymmetric, exact confidence intervals.

 \item {\bf DKW and normal approximations} are useful for quick, rough, or uniform (simultaneous) bands but they are conservative and/or inappropriate at the extremes. They do not contradict the binomial/Beta facts — they are simply different (and usually looser) guarantees.
\end{itemize}

\subsection*{Full Explanation}

\subsubsection*{1. Why the normal approximation fails}

The ECDF lies in \([0,1]\), is discrete for finite \(N\), and is skewed near 0 and 1. A normal CI can exceed the unit interval.

\subsubsection*{2. Correct distribution at fixed \(x\)}

If \(k\) points fall \(\le x\), then \(k \sim \mathrm{Binomial}(N, F(x))\).

Exact Clopper--Pearson CI:

\[
p_L = \mathrm{Beta}^{-1}(\alpha/2; k, N-k+1), \quad
p_U = \mathrm{Beta}^{-1}(1-\alpha/2; k+1, N-k).
\]

\subsubsection*{3. Correct distribution for order statistics}

\[
U_i = F(X_{(i)}) \sim \mathrm{Beta}(i, N+1-i),
\]

with variances matching Gumbel's formula.

The CI is:

\[
\left(
  \mathrm{Beta}^{-1}\left(\frac{\alpha}{2}; i, N+1-i \right),
  \mathrm{Beta}^{-1}\left(1-\frac{\alpha}{2}; i, N+1-i \right)
\right).
\]

\subsubsection*{4. Uniform (DKW) bands}

DKW gives a distribution-free bound:

\[
\sup_x |\hat F_N(x) - F(x)| \le \sqrt{\frac{\ln(2/\alpha)}{2N}}.
\]

Useful for full-range coverage but conservative.

\subsection*{Practical Recommendations}

\begin{itemize}
  \item Use Clopper--Pearson, Wilson, or Jeffreys intervals for fixed \(x\)
  \item Use Beta quantiles for plotting positions
  \item Use DKW only for uniform bands
\end{itemize}

\subsection*{Final Conclusion}

The appropriate distributions for ECDF confidence intervals are:

\[
\text{Binomial (fixed x)} \qquad\text{and}\qquad \text{Beta (order statistics)}.
\]

These provide proper bounded, asymmetric, exact intervals.
\bigskip

\end{document}
